\section{Conclusion}
\label{sec:conclusion}
In this paper, we introduce two effective semi-supervised approaches for exploiting unlabeled frames in videos to improve semantic segmentation performance of indoor scenes. 
%In order to make full use of unlabeled samples, 
First, we propagate the labeled data to neighboring frames and generate a large number of reliable pseudo labels to enrich the training set.
%Despite the existence of noisy labels in PGT, the network can still learn more abundant information from PGT.
%
Second, our training policy takes advantage of temporal correlations between adjacent frames to enhance the semantic segmentation performance.
%
Experimental results on NYUD-v2 dataset demonstrate the superiority of our proposed method.  
%{\bf Future Work.} As we said in the discussion, the imbalance of sample classes results in the error of segmentation results. 
%
%Hence, our future work is still based on the data itself.
%
%One direction is how to balance sample categories, and the other is how to make more efficient and reasonable use of unlabeled video data.

