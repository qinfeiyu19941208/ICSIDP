\begin{abstract}
Semantic segmentation is a fundamental task in indoor scene understanding. 
Most previous supervised approaches rely on densely annotated image datasets. 
Due to the limited amount of images with segmentation labels, the performance and robustness of existing networks is greatly limited.
In this work, we exploit spatical and temporal correlation in video frames to improve the performance of segmentation networks. 
Two effective semi-supervised learning strategies are proposed to propagate the information from a few labeled frames to adjacent video frames. 
First, we augment training data for supervised semantic segmentation networks by generating pseudo ground-truth for neighboring frames from a labeled frame using filtered homography transformation.
Second, we introduce a self-supervised loss function to ensure temporal consistency between the segmentation results of adjacent frames. 
The experimental results show that our proposed method outperforms state-of-the-art techniques for semantic segmentation on NYU-Depth V2 dataset.

\end{abstract}
\begin{IEEEkeywords}
	Indoor scene, semantic segmentation, temporal consistency, label propagetion 
\end{IEEEkeywords}